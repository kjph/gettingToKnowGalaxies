%=== PACKAGES ==================================================================
  \documentclass[a4paper]{article}
  %Appearance
  \usepackage{microtype} %better typography
  \usepackage[english]{babel}
  \usepackage{fancyhdr} %for fancy page style
  \usepackage{color} %Added support for color text
  \usepackage{dsfont}

  %Document Dynamics
  \usepackage[colorlinks=true]{hyperref} %Manage links
  \hypersetup{linkcolor=blue}
  \usepackage{url}
  \usepackage{geometry} %For easy management of document margins
  \usepackage{graphicx} %Allows for insertion of images in document
  \graphicspath{ {figures/} {images/} }

  %Technical
  \usepackage{amsmath}
  \usepackage{amsthm}
  \usepackage{amssymb}
  \usepackage{tikz}
  \usepackage{listings} %To insert code blocks

%=== TITLE =====================================================================
  \title{Determining Semi-Major Axis Scalar for Ellipse passing through SN}
  \author{}
  \date{Last Updated: \today}

%=== FRONT MATTER ==============================================================
  \begin{document}
  \pagestyle{fancy} \lhead{} \rhead{}
  \maketitle

%=== BODY ======================================================================
\noindent We first determine the relative position ($\Delta\vec{d}_{RD}$) of the center of the supernovae,
$\vec{S}_{RD}$, in terms of the center of the galaxy, $\vec{G}_{RD}$, where `RD'
implies the right acsencion, declination coordinates (in degrees).
    \begin{equation}
    \Delta\vec{d}_{RD} = \vec{S}_{RD} - \vec{G}_{RD}\
    \end{equation}

\noindent We then convert this relative position vector in terms of pixel coordinates `XY'
    \begin{equation}
    \Delta\vec{d}_{XY} = \mathbf{T_1}\cdot\Delta\vec{d}_{RD}\\
    \end{equation}

\noindent Where $\mathbf{T_1}$ is the scalar transformation, with $P$ the
arcsecond--pixel ratio;
    \begin{equation}
    \mathbf{T_1} = \frac{1}{3600P}\cdot\mathds{1}\\
    \end{equation}

\noindent We can then determine the supernovae's position in terms of the semi-major and
semi-minor axies through a rotation matrix;
\begin{subequations}
    \begin{align}
    \begin{pmatrix}b\\a\end{pmatrix}_{SN}=\mathbf{T_2}\cdot\Delta\vec{d}_{XY}\\
    \mathbf{T_2} = \begin{pmatrix}\cos\theta&-\sin\theta\\ R\sin\theta&R\cos\theta\end{pmatrix}
    \end{align}
\end{subequations}

\noindent Where $\theta$ is the position angle and $R$ is the axial ratio

\end{document}

